% Options for packages loaded elsewhere
\PassOptionsToPackage{unicode}{hyperref}
\PassOptionsToPackage{hyphens}{url}
\PassOptionsToPackage{dvipsnames,svgnames,x11names}{xcolor}
%
\documentclass[
  11.5pt,
  leqno]{scrartcl}

\usepackage{amsmath,amssymb}
\usepackage{setspace}
\usepackage{iftex}
\ifPDFTeX
  \usepackage[T1]{fontenc}
  \usepackage[utf8]{inputenc}
  \usepackage{textcomp} % provide euro and other symbols
\else % if luatex or xetex
  \usepackage{unicode-math}
  \defaultfontfeatures{Scale=MatchLowercase}
  \defaultfontfeatures[\rmfamily]{Ligatures=TeX,Scale=1}
\fi
\usepackage{lmodern}
\ifPDFTeX\else  
    % xetex/luatex font selection
\fi
% Use upquote if available, for straight quotes in verbatim environments
\IfFileExists{upquote.sty}{\usepackage{upquote}}{}
\IfFileExists{microtype.sty}{% use microtype if available
  \usepackage[]{microtype}
  \UseMicrotypeSet[protrusion]{basicmath} % disable protrusion for tt fonts
}{}
\makeatletter
\@ifundefined{KOMAClassName}{% if non-KOMA class
  \IfFileExists{parskip.sty}{%
    \usepackage{parskip}
  }{% else
    \setlength{\parindent}{0pt}
    \setlength{\parskip}{6pt plus 2pt minus 1pt}}
}{% if KOMA class
  \KOMAoptions{parskip=half}}
\makeatother
\usepackage{xcolor}
\setlength{\emergencystretch}{3em} % prevent overfull lines
\setcounter{secnumdepth}{-\maxdimen} % remove section numbering


\providecommand{\tightlist}{%
  \setlength{\itemsep}{0pt}\setlength{\parskip}{0pt}}\usepackage{longtable,booktabs,array}
\usepackage{calc} % for calculating minipage widths
% Correct order of tables after \paragraph or \subparagraph
\usepackage{etoolbox}
\makeatletter
\patchcmd\longtable{\par}{\if@noskipsec\mbox{}\fi\par}{}{}
\makeatother
% Allow footnotes in longtable head/foot
\IfFileExists{footnotehyper.sty}{\usepackage{footnotehyper}}{\usepackage{footnote}}
\makesavenoteenv{longtable}
\usepackage{graphicx}
\makeatletter
\def\maxwidth{\ifdim\Gin@nat@width>\linewidth\linewidth\else\Gin@nat@width\fi}
\def\maxheight{\ifdim\Gin@nat@height>\textheight\textheight\else\Gin@nat@height\fi}
\makeatother
% Scale images if necessary, so that they will not overflow the page
% margins by default, and it is still possible to overwrite the defaults
% using explicit options in \includegraphics[width, height, ...]{}
\setkeys{Gin}{width=\maxwidth,height=\maxheight,keepaspectratio}
% Set default figure placement to htbp
\makeatletter
\def\fps@figure{htbp}
\makeatother
% definitions for citeproc citations
\NewDocumentCommand\citeproctext{}{}
\NewDocumentCommand\citeproc{mm}{%
  \begingroup\def\citeproctext{#2}\cite{#1}\endgroup}
\makeatletter
 % allow citations to break across lines
 \let\@cite@ofmt\@firstofone
 % avoid brackets around text for \cite:
 \def\@biblabel#1{}
 \def\@cite#1#2{{#1\if@tempswa , #2\fi}}
\makeatother
\newlength{\cslhangindent}
\setlength{\cslhangindent}{1.5em}
\newlength{\csllabelwidth}
\setlength{\csllabelwidth}{3em}
\newenvironment{CSLReferences}[2] % #1 hanging-indent, #2 entry-spacing
 {\begin{list}{}{%
  \setlength{\itemindent}{0pt}
  \setlength{\leftmargin}{0pt}
  \setlength{\parsep}{0pt}
  % turn on hanging indent if param 1 is 1
  \ifodd #1
   \setlength{\leftmargin}{\cslhangindent}
   \setlength{\itemindent}{-1\cslhangindent}
  \fi
  % set entry spacing
  \setlength{\itemsep}{#2\baselineskip}}}
 {\end{list}}
\usepackage{calc}
\newcommand{\CSLBlock}[1]{\hfill\break\parbox[t]{\linewidth}{\strut\ignorespaces#1\strut}}
\newcommand{\CSLLeftMargin}[1]{\parbox[t]{\csllabelwidth}{\strut#1\strut}}
\newcommand{\CSLRightInline}[1]{\parbox[t]{\linewidth - \csllabelwidth}{\strut#1\strut}}
\newcommand{\CSLIndent}[1]{\hspace{\cslhangindent}#1}

\usepackage[sfdefault]{roboto}
\usepackage{unicode-math}
\usepackage[a4paper,headheight=13pt,footskip=20pt,margin=2.54cm]{geometry}
\usepackage{indentfirst}

% Modify section headings
\usepackage{titlesec}
\titleformat{\section}{\normalfont\large\bfseries\centering}{}{.5em}{}
\titleformat{\subsection}{\normalfont\bfseries}{}{.5em}{}
\titleformat{\subsubsection}[runin]{\normalfont\itshape}{\thesubsubsection.}{.5em}{\addperiod}
\newcommand{\addperiod}[1]{#1.} %https://tex.stackexchange.com/questions/137193/adding-a-period-after-section

\setlength{\parindent}{1.5em}
\setlength{\parskip}{0pt plus 1pt}
\linespread{1.1}

% Adjust caption style
\usepackage[labelfont=bf,
    labelsep=endash,
    singlelinecheck=off,
    format=plain,
    margin=1.5cm,
    aboveskip=12pt,
    belowskip=12pt,
    font=footnotesize,
    justification=justified]{caption}

% Defining document colors
\usepackage{xcolor}
\definecolor{darkgray}{HTML}{808080}
\definecolor{mediumgray}{HTML}{6D6E70}
\definecolor{ligthgray}{HTML}{d9d9d9}
\definecolor{pciblue}{HTML}{346E92}
\definecolor{opengreen}{HTML}{77933c}

%\usepackage[colorlinks=true,linkcolor=pciblue,urlcolor=pciblue,citecolor=pciblue]{hyperref} % Not working - links in pdf not changing colour. 
                                                         %Probably because hyperref is no the last package to be loaded, which it needs to be.
%\urlstyle{same}
\usepackage{booktabs}
\usepackage{longtable}
\usepackage{array}
\usepackage{multirow}
\usepackage{wrapfig}
\usepackage{float}
\usepackage{colortbl}
\usepackage{pdflscape}
\usepackage{tabu}
\usepackage{threeparttable}
\usepackage{threeparttablex}
\usepackage[normalem]{ulem}
\usepackage{makecell}
\usepackage{xcolor}
\makeatletter
\@ifpackageloaded{caption}{}{\usepackage{caption}}
\AtBeginDocument{%
\ifdefined\contentsname
  \renewcommand*\contentsname{Table of contents}
\else
  \newcommand\contentsname{Table of contents}
\fi
\ifdefined\listfigurename
  \renewcommand*\listfigurename{List of Figures}
\else
  \newcommand\listfigurename{List of Figures}
\fi
\ifdefined\listtablename
  \renewcommand*\listtablename{List of Tables}
\else
  \newcommand\listtablename{List of Tables}
\fi
\ifdefined\figurename
  \renewcommand*\figurename{Figure}
\else
  \newcommand\figurename{Figure}
\fi
\ifdefined\tablename
  \renewcommand*\tablename{Table}
\else
  \newcommand\tablename{Table}
\fi
}
\@ifpackageloaded{float}{}{\usepackage{float}}
\floatstyle{ruled}
\@ifundefined{c@chapter}{\newfloat{codelisting}{h}{lop}}{\newfloat{codelisting}{h}{lop}[chapter]}
\floatname{codelisting}{Listing}
\newcommand*\listoflistings{\listof{codelisting}{List of Listings}}
\makeatother
\makeatletter
\makeatother
\makeatletter
\@ifpackageloaded{caption}{}{\usepackage{caption}}
\@ifpackageloaded{subcaption}{}{\usepackage{subcaption}}
\makeatother
\ifLuaTeX
  \usepackage{selnolig}  % disable illegal ligatures
\fi
\usepackage{bookmark}

\IfFileExists{xurl.sty}{\usepackage{xurl}}{} % add URL line breaks if available
\urlstyle{same} % disable monospaced font for URLs
\hypersetup{
  pdftitle={Multiproxy analysis exploring patterns of diet and disease in dental calculus and skeletal remains from a 19th century Dutch population},
  pdfauthor={Bjørn Peare Bartholdy; Jørgen B. Hasselstrøm; Lambert K. Sørensen; Maia Casna; Menno Hoogland; Historisch Genootschap Beemster; Amanda G. Henry},
  pdfkeywords={dental calculus; LC-MS/MS; alkaloids; dental pathology;
sinusitis; caffeine; tobacco},
  colorlinks=true,
  linkcolor={pciblue},
  filecolor={Maroon},
  citecolor={pciblue},
  urlcolor={pciblue},
  pdfcreator={LaTeX via pandoc}}

\title{}
\author{}
\date{}

\begin{document}

\setstretch{1.5}
\section{Introduction}\label{introduction}

Dental calculus has proven to be an excellent source of a wide variety
of information about our past. The increased accessibility and
advancement of methods in aDNA, paleoproteomics, and mass spectrometry,
has expanded our ability to identify biomarkers of diet and disease on
an increasingly large scale
(\citeproc{ref-gismondiMultidisciplinaryApproach2020}{Gismondi et al.,
2020}; \citeproc{ref-velskoDentalCalculus2017}{Velsko et al., 2017};
\citeproc{ref-warinnerEvidenceMilk2014}{Warinner et al., 2014}).

One such collection of biomarkers is alkaloids, a plant-derived group of
compounds. Many alkaloids have important medicinal and psychoactive
effects in humans, and their direct detection, or detection of their
metabolites, is of great interest to archaeologists. Previous studies
have successfully recovered alkaloids in archaeological contexts,
including ceramics (\citeproc{ref-smithDetectionOpium2018}{Smith et al.,
2018}), pipes (\citeproc{ref-raffertyCurrentResearch2012}{Rafferty et
al., 2012}), human hair
(\citeproc{ref-echeverriaNicotineHair2013}{Echeverría \& Niemeyer,
2013}; \citeproc{ref-ogaldeIdentificationPsychoactive2009}{Ogalde et
al., 2009}), and even dental calculus employing both targeted
(\citeproc{ref-eerkensDentalCalculus2018}{Eerkens et al., 2018}) and
untargeted approaches (\citeproc{ref-buckleyDentalCalculus2014}{Buckley
et al., 2014};
\citeproc{ref-gismondiMultidisciplinaryApproach2020}{Gismondi et al.,
2020}). Especially nicotine, the principal alkaloid in tobacco leaves,
has been widely studied in the archaeological record due to its apparent
stability and ability to survive over long periods of time
(\citeproc{ref-eerkensDentalCalculus2018}{Eerkens et al., 2018};
\citeproc{ref-raffertyCurrentResearch2012}{Rafferty et al., 2012};
\citeproc{ref-tushinghamHuntergathererTobacco2013}{Tushingham et al.,
2013}).

Alkaloids may enter the oral cavity via two pathways: (1) direct
incorporation through ingestion of alkaloid-containing plants, whether
deliberate or accidental; and (2) passive diffusion as alkaloids and
other compounds are transferred from plasma to saliva, and then
gradually secreted into the oral cavity through the salivary glands in
the hours-to-days following ingestion
(\citeproc{ref-coneInterpretationOral2007}{Cone \& Huestis, 2007}). The
second pathway allows the identification of parent compounds that do not
enter the mouth (e.g.~injection), as long as they, or their metabolites,
are excreted through the saliva, thus eventually entering the oral
cavity.

Many of the components involved in the formation and growth of dental
calculus originate from oral fluid. Proteins, bacteria, salts and other
compounds are transferred from saliva to biofilms on the tooth surface
(\citeproc{ref-jinSupragingivalCalculus2002}{Jin \& Yip, 2002};
\citeproc{ref-whiteDentalCalculus1997}{White, 1997}). This may also
allow various alkaloids of dietary and medicinal origin to become
incorporated in dental plaque. Dental plaque undergoes frequent
mineralisation events, ultimately causing the entrapped alkaloids and
their metabolites to become preserved within the dental calculus.
Barring intentional or accidental removal of the calculus during life,
burial, excavation, and post-excavation cleaning, the alkaloids can then
be detected by various methods to show a record of consumption during
life. Because drugs may be transferred from plasma to saliva, there is
often a close correlation between drugs detected in oral fluid and
blood, though there are differences in detected concentrations
(\citeproc{ref-coneInterpretationOral2007}{Cone \& Huestis, 2007};
\citeproc{ref-milmanOralFluid2011}{Milman et al., 2011};
\citeproc{ref-willeRelationshipOral2009}{Wille et al., 2009}). This was
also shown to be true for dental calculus and blood
(\citeproc{ref-sorensenDrugsCalculus2021}{Sørensen et al., 2021}),
making dental calculus a potentially useful substance for detecting
ancient alkaloids and other dietary compounds.

In this study we use a ultra-high-performance liquid
chromatography-tandem mass spectrometry (UHPLC-MS/MS) method that was
developed in a previous study on dental calculus from cadavers and
validated by comparing the results to compounds detected in the blood of
the same individuals (\citeproc{ref-sorensenDrugsCalculus2021}{Sørensen
et al., 2021}). All compounds that were detected in the blood were also
detected in dental calculus, with additional compounds present in dental
calculus that were not present in blood, suggesting that dental calculus
represents a comprehensive history of consumption over a long period of
time (\citeproc{ref-sorensenDrugsCalculus2021}{Sørensen et al., 2021}).
We were able to detect both parent compounds and metabolites, including
caffeine, nicotine, theophylline, and cotinine, in the dental calculus
of individuals from a 19th century Dutch population from Middenbeemster.
By detecting these compounds we are able to show the consumption of tea
and coffee and smoking of tobacco on an individual scale, which is also
confirmed by historic documentation and identification of pipe notches
in the dentition.

\section{Materials}\label{materials}

The sample consists of 41 individuals from Middenbeemster, a 19th
century rural Dutch site. The village of Middenbeemster and the
surrounding Beemsterpolder was established in the beginning of the 17th
century, when the Beemster lake was drained to create more farmland,
mainly for the cultivation of cole seeds (de Vries 1978). In 1615, a
decision was made to build a church, and construction started in 1618
(Hakvoort 2013). The excavated cemetery is associated with the
Keyserkerk church, where the inhabitants of the Middenbeemster village
and the surrounding Beemsterpolder were buried between AD 1615 and 1866
(\citeproc{ref-lemmersMiddenbeemster2013}{Lemmers et al., 2013}).
Archival documents are available for those buried between AD 1829 and
1866, when the majority of individuals were interred. The main
occupation of the inhabitants was dairy farming, consisting largely of
manual labour prior to the industrial revolution
(\citeproc{ref-aten400Jaar2012}{Aten et al., 2012};
\citeproc{ref-palmerActivityReconstruction2016}{Palmer et al., 2016}).

For our sample, we preferentially selected males from the middle adult
age category (35-49 years) to minimise the effect of confounding
cultural and biological factors. Previous research on Middenbeemster has
shown a gendered division of labour
(\citeproc{ref-palmerActivityReconstruction2016}{Palmer et al., 2016}),
and there are biological differences in dental calculus formation and
drug metabolism that are related to age and sex
(\citeproc{ref-huangDecipheringGenetic2023}{Huang et al., 2023};
\citeproc{ref-unoSexAgedependent2017}{Uno et al., 2017};
\citeproc{ref-whiteDentalCalculus1997}{White, 1997}). The sample
consists of 27 males, 11 probable males, 2 probable females, and 1
female (Figure~\ref{fig-sample-demography}). We selected males due to a
higher occurrence of pipe notches and dental calculus deposits than
females (unpublished observation).

\begin{figure}

\centering{

\includegraphics{../figures/fig-sample-demography-1.pdf}

}

\caption{\label{fig-sample-demography}Overview of sample demography.
Left plot is the first batch and right plot is the replication batch
with 29 of the individuals from the first batch. eya = early young adult
(18-24 years); lya = late young adult (25-34 years); ma = middle adult
(35-49 years); old = old adult (50+ years). Male? = probable male;
Female? = probable female.}

\end{figure}%

\section{Methods}\label{methods}

\subsection{Skeletal analysis}\label{skeletal-analysis}

Demographic and pathological analyses were conducted in the Laboratory
for Human Osteoarchaeology at Leiden University. Sex was estimated using
cranial and pelvic morphological traits
(\citeproc{ref-Standards1994}{Buikstra \& Ubelaker, 1994}). Age-at-death
was estimated using dental wear, auricular and pubic surface appearance,
cranial suture closure, and epiphyseal fusion
(\citeproc{ref-SucheyBrooks1990}{Brooks \& Suchey, 1990};
\citeproc{ref-buckberryAuricular2002}{Buckberry \& Chamberlain, 2002};
\citeproc{ref-Standards1994}{Buikstra \& Ubelaker, 1994};
\citeproc{ref-lovejoyAuricular1985}{Lovejoy et al., 1985};
\citeproc{ref-meindlSutureClosure1985}{Meindl \& Lovejoy, 1985}), and
divided into the following categories: early young adult (18-24 years),
late young adult (25-34 years), middle adult ( 35-49 years), old adult
(50+ years). Preservation was visually scored on a four-stage scale
(excellent, good, fair, poor) based on the surface condition of the
bones and the extent of taphonomic degradation.

\subsubsection{Paleopathology}\label{paleopathology}

Pathological conditions and lesions that occur frequently in the
population were included in the analysis. Data were dichotomised to
presence/absence to allow for statistical analysis. Osteoarthritis was
considered present in cases where eburnation was visible on one or more
joint surfaces. Vertebral osteophytosis is identified by marginal
lipping and/or osteophyte formation on the margin of the superior and
inferior surfaces of the vertebral body. Cribra orbitalia was diagnosed
based on the presence of pitting on the superior surface of the orbit.
No distinction was made between active or healing lesions. Degenerative
disc disease, or spondylosis, is identified as a large diffuse
depression of the superior and/or inferior surfaces of the vertebral
body (\citeproc{ref-rogersPalaeopathologyJoint2000}{Rogers, 2000}).
Schmorl's nodes are identified as any cortical depressions on the
surface of the vertebral body. Data on chronic maxillary sinusitis from
Casna et al. (\citeproc{ref-casnaUrbanizationRespiratory2021}{2021})
were included in this study to assess the relationship between upper
respiratory diseases with environmental factors (i.e.~tobacco smoke,
caffeine consumption). Lesions associated with chronic maxillary
sinusitis as defined by Boocock et al.
(\citeproc{ref-boocockMaxillarySinusitis1995}{1995}) were recorded for
each individual and classified as ``pitting'', ``spicule-type bone
formation'', ``remodeled spicules'', or ``white pitted bone''. chronic
maxillary sinusitis was scored as absent when the sinus presented smooth
surfaces with little or no associated pitting.

\subsubsection{Dental pathology}\label{dental-pathology}

Caries ratios were calculated by dividing the number of lesions by the
number of teeth scored, resulting in a single caries ratio per
individual. If the surface where the lesion originated is not visible,
i.e.~if the lesion covered multiple surfaces, this was scored as
``crown''. Calculus indices were calculated according to Greene and
colleagues (\citeproc{ref-greeneQuantifyingCalculus2005}{2005}).
Calculus was scored with a four-stage scoring system (0-3) to score
absent, slight, moderate, and heavy calculus deposits
(\citeproc{ref-brothwellDiggingBones1981}{Brothwell, 1981}) on the
lingual, buccal (and labial), and interproximal surfaces of each tooth.
Only one score was used for the combined interproximal surfaces,
resulting in three scores per tooth (when surfaces are intact), and four
calculus indices per individual; upper anterior, upper posterior, lower
anterior, lower posterior. Each index was calculated by dividing the sum
of calculus scores for each surface by the total number of surfaces
scored in each quadrant. If a tooth could not be scored on all three
surfaces, the tooth was not included
(\citeproc{ref-greeneQuantifyingCalculus2005}{Greene et al., 2005}).
Periodontitis was scored on a visual four-stage (0-3) scoring system
according to distance from cemento-enamel junction of each tooth to
alveolar bone (\citeproc{ref-maatManualPhysical2005}{Maat \& Mastwijk,
2005}).

\subsection{Calculus sampling}\label{calculus-sampling}

Where possible, we used material that had already been sampled for a
previous study to prevent unnecessary repeated sampling of individuals.
Calculus from the previous study was sampled in a dedicated ancient DNA
laboratory at the Laboratories of Molecular Anthropology and Microbiome
Research in Norman, Oklahoma, U.S.A, using established ancient DNA
protocols. More details on the methods can be found in the published
articles (\citeproc{ref-ziesemer16SChallenges2015}{Ziesemer et al.,
2015}, \citeproc{ref-ziesemerGenomeCalculus2018}{2018}). Of the 41
individuals that were originally included in our sample, 29 were
replicated in a separate analysis only using calculus from the previous
study.\\
New dental calculus samples were taken under sterile conditions in a
positive pressure laminar flow hood in a dedicated dental calculus lab
at Leiden University. The surface of the tooth was lightly brushed with
a sterile, disposable toothbrush to get rid of surface contaminants. A
sterile dental curette was then used to scrape calculus from the tooth
onto weighing paper, which was transferred to 1.5 ml Eppendorf tubes.
All calculus samples were sent to the Department of Forensic Medicine at
Aarhus University for ultra-high-performance liquid
chromatography-tandem mass spectrometry (UHPLC-MS/MS) analysis.

\subsection{UHPLC-MS/MS}\label{uhplc-msms}

The list of targeted compounds included both naturally occurring
compounds known to have been used in the past, as well as synthetic
modern drugs that did not exist at the time (e.g.~Fentanyl, MDMA,
Amphetamine). These were part of the toxicology screening for the
original method (\citeproc{ref-sorensenDrugsCalculus2021}{Sørensen et
al., 2021}), developed on cadavers. In our study they serve as an
authentication step, as their presence in archaeological samples could
only be the result of contamination.

Briefly, samples of dental calculus were washed three times each with
one mL of methanol (MeOH), to remove surface contaminants. The wash
solutions were collected separately. The solvent was evaporated and the
residues were dissolved in 50 µL 30\% MeOH. The washed calculus was
homogenized in presence of 0.5 M citric acid using a lysing tube with
stainless steel beads. Following one hour of incubation the dissolution
extract was cleaned by weak and strong cation-exchange. After
evaporation of the elution solvent the residue was dissolved in 50 µL
30\% MeOH. The final extracts obtained from washing and dissolution of
the dental calculus were analysed by UHPLC-MS/MS using a reversed-phase
biphenyl column for chromatography. To obtain quantitative results,
isotope dilution was applied. For more details about the method and
validation, see the original study by Sørensen and colleagues
(\citeproc{ref-sorensenDrugsCalculus2021}{2021}).

\subsection{Statistical analysis}\label{statistical-analysis}

All compounds and pathological conditions/lesions were converted to a
presence/absence score. Pearson product-moment correlation was applied
to the dichotomised pathological lesions (point-biserial correlation),
compound concentrations, calculus indices, and caries ratios to explore
relationships paired continuous-continuous variables and paired
continuous-binary variables. Compound concentrations were then
dichotomised to presence/absence, and the caries ratio and calculus
index for each individual were converted to an ordinal score from 0 to 4
by using quartiles. Polychoric correlation was applied to the paired
dichotomous variables and dichotomous-ordinal variables.

All statistical analysis was conducted in R version 4.3.3 (2024-02-29),
Angel Food Cake, (\citeproc{ref-Rbase}{R Core Team, 2020}). Data
wrangling was conducted with the \textbf{tidyverse}
(\citeproc{ref-tidyverse2019}{Wickham et al., 2019}) and visualisations
were created using \textbf{ggplot2} (\citeproc{ref-ggplot2}{Wickham,
2016}). Polychoric correlations were calculated with the \textbf{psych}
package (\citeproc{ref-Rpsych}{Revelle, 2022}).

\section{Results}\label{results}

Multiple compounds were detected in the dental calculus samples.
Compounds detected at a lower concentration than the lower limit of
quantitation (LLOQ) were considered not present. Not all the compounds
detected in the first batch could be replicated in the second batch
(Table~\ref{tbl-compound-detect}). For a full list of targeted
compounds, see Supplementary Material.

\begin{longtable}[]{@{}lllr@{}}

\caption{\label{tbl-compound-detect}Target compound including whether it
was detected (TRUE) or not (FALSE) in each batch, as well as the lower
limit of quantitation (LLOQ) in ng. CBD = cannabidiol; CBN = cannabinol;
THC = tetrahydrocannabinol; THCA-A = tetrahydrocannabinolic acid A;
THCVA = tetrahydrocannabivarin acid.}

\tabularnewline

\toprule\noalign{}
Compound & Batch 1 & Batch 2 & LLOQ \\
\midrule\noalign{}
\endhead
\bottomrule\noalign{}
\endlastfoot
CBD & TRUE & FALSE & 0.050 \\
CBN & TRUE & FALSE & 0.050 \\
Caffeine & TRUE & TRUE & 0.050 \\
Cocaine & TRUE & FALSE & 0.025 \\
Cotinine & TRUE & TRUE & 0.050 \\
Nicotine & TRUE & TRUE & 0.100 \\
Salicylic acid & TRUE & TRUE & 0.500 \\
THC & TRUE & FALSE & 0.100 \\
THCA-A & TRUE & FALSE & 0.025 \\
THCVA & TRUE & FALSE & 0.010 \\
Theophylline & TRUE & TRUE & 0.010 \\

\end{longtable}

The pattern we expect to see in authentic compounds representing
compounds trapped within the dental calculus, is a reduction in the
quantity from wash 1 to wash 3 as potential surface contaminants are
washed off, and then a spike in the final extraction when entrapped
compounds are released and detected.

Most plots show a large increase in extracted mass of a compound between
the calculus wash extracts (wash 1-3) and the dissolved calculus (calc).
Most samples containing theophylline and caffeine had the largest
quantity of the compound extracted from the first wash, then decreasing
in washes 2 and 3. There is an increase between wash 3 and the dissolved
calculus in all samples. The patterns are consistent across batches 1
and 2. Nicotine and cotinine have the same relative quantities in the
samples, i.e., the sample with the highest extracted quantity of
nicotine also had the highest extracted quantity of cotinine
(Figure~\ref{fig-auth-plot-batch2}).

\begin{figure}

\centering{

\includegraphics{../figures/fig-auth-plot-batch2-1.pdf}

}

\caption{\label{fig-auth-plot-batch2}(A) Number of samples in which each
compound was detected in the first and second batch. (B) Quantity (ng)
of each compound extracted from each sample in batch 2. The plot
displays the extracted quantity across the three washes and final
calculus extraction (calc). Each coloured line represents a different
calculus sample. CBD = cannabidiol; CBN = cannabinol; THC =
tetrahydrocannabinol; THCA-A = tetrahydrocannabinolic acid A; THCVA =
tetrahydrocannabivarin acid.}

\end{figure}%

To see if preservation of the skeletal remains had any effect on the
detection of compounds, we compare extracted quantities of compounds to
the various levels of skeletal preservation. Our results from batch 2
suggest that detection of a compound may be linked to the preservation
of the skeleton, with better preservation leading to increased
extraction quantity (Figure~\ref{fig-detection-preservation}A). We also
find a weak positive correlation between the weight of the calculus
sample and the quantity of compound extracted from the calculus
(Figure~\ref{fig-detection-preservation}B).

\begin{figure}

\centering{

\includegraphics{../figures/fig-detection-preservation-1.pdf}

}

\caption{\label{fig-detection-preservation}(A) Violin plot with overlaid
box plots depicting the distribution of extracted quantities of each
compound from batch 2 separated by state of preservation of the
skeleton. (B) Extracted quantity (ng) of compound plotted against
weights of the calculus samples from batch 2. r = Pearson correlation
coefficient.}

\end{figure}%

The presence of pipe notch(es) in an individual and concurrent detection
of nicotine and/or cotinine is used as a crude indicator of the accuracy
of the method. Only males were used in accuracy calculations, as pipe
notches are ubiquitous in males, but not in females. In batch 2, the
method was able to detect some form of tobacco in 14 of 25 individuals
with a pipe notch (56.0\%). When also considering correct the absence of
a tobacco alkaloid together with the absence of a pipe notch, the
accuracy of the method is 59.3\%. Accuracy in the old adult age category
is 100.0\%, but with only 2 individuals.

One individual---an old adult, probable female---was positive for both
nicotine and cotinine, and had no signs of a pipe notch.

\subsection{Correlations between detected alkaloids and
diseases}\label{correlations-between-detected-alkaloids-and-diseases}

For further statistical analyses, only the UHPLC-MS/MS results from
batch 2 were used, as batch 1 had multiple compounds that were not
detected in batch 2 and may have been contaminated.

\begin{longtable}[]{@{}
  >{\raggedright\arraybackslash}p{(\columnwidth - 16\tabcolsep) * \real{0.1413}}
  >{\raggedright\arraybackslash}p{(\columnwidth - 16\tabcolsep) * \real{0.0761}}
  >{\raggedright\arraybackslash}p{(\columnwidth - 16\tabcolsep) * \real{0.0978}}
  >{\raggedright\arraybackslash}p{(\columnwidth - 16\tabcolsep) * \real{0.0652}}
  >{\raggedright\arraybackslash}p{(\columnwidth - 16\tabcolsep) * \real{0.0978}}
  >{\raggedright\arraybackslash}p{(\columnwidth - 16\tabcolsep) * \real{0.0652}}
  >{\raggedright\arraybackslash}p{(\columnwidth - 16\tabcolsep) * \real{0.1522}}
  >{\raggedright\arraybackslash}p{(\columnwidth - 16\tabcolsep) * \real{0.1522}}
  >{\raggedright\arraybackslash}p{(\columnwidth - 16\tabcolsep) * \real{0.1522}}@{}}

\caption{\label{tbl-pearson}Pearson correlation (\emph{r}) on
dichotomous skeletal lesions and compound concentrations (ng/mg) from
the second batch. Correlations between pairs of dichotomous variables
are removed due to incompatibility with a Pearson correlation. Moderate
and strong correlations in \textbf{bold}. OA = osteoarthritis; VOP =
vertebral osteophytosis; SN = Schmorl's nodes; DDD = degenerative disc
disease; CO = cribra orbitalia; CMS = chronic maxillary sinusitis; SA =
salicylic acid; PN = pipe notches.}

\tabularnewline

\toprule\noalign{}
\begin{minipage}[b]{\linewidth}\raggedright
\end{minipage} & \begin{minipage}[b]{\linewidth}\raggedright
Caries
\end{minipage} & \begin{minipage}[b]{\linewidth}\raggedright
Nicotine
\end{minipage} & \begin{minipage}[b]{\linewidth}\raggedright
SA
\end{minipage} & \begin{minipage}[b]{\linewidth}\raggedright
Calculus
\end{minipage} & \begin{minipage}[b]{\linewidth}\raggedright
PN
\end{minipage} & \begin{minipage}[b]{\linewidth}\raggedright
Theophylline
\end{minipage} & \begin{minipage}[b]{\linewidth}\raggedright
Caffeine
\end{minipage} & \begin{minipage}[b]{\linewidth}\raggedright
Cotinine
\end{minipage} \\
\midrule\noalign{}
\endhead
\bottomrule\noalign{}
\endlastfoot
OA & -0.12 & -0.07 & 0.21 & 0.07 & 0.14 & 0.28 & 0 & -0.07 \\
VOP & -0.09 & -0.16 & 0.34 & 0.06 & 0.25 & -0.06 & 0.01 & -0.13 \\
SN & -0.24 & 0.16 & 0.09 & 0.09 & 0.17 & 0.24 & 0.16 & 0.09 \\
DDD & 0 & 0 & 0.19 & -0.39 & -0.08 & 0.31 & 0.06 & -0.01 \\
CO & 0.06 & -0.05 & 0.2 & 0.14 & -0.2 & -0.11 & 0.19 & -0.06 \\
CMS & -0.19 & 0.28 & 0 & -0.27 & 0.03 & 0.19 & 0.36 & 0.22 \\
Caries & & -0.2 & -0.36 & -0.15 & -0.17 & -0.21 & 0 & -0.22 \\
Nicotine & & & -0.21 & 0.01 & -0.01 & \textbf{0.43} & 0.14 &
\textbf{0.98} \\
SA & & & & 0.14 & 0.37 & 0.04 & 0.17 & -0.17 \\
Calculus & & & & & 0.13 & -0.15 & -0.13 & 0.03 \\
PN & & & & & & -0.16 & 0.18 & -0.01 \\
Theophylline & & & & & & & \textbf{0.51} & 0.36 \\
Caffeine & & & & & & & & 0.08 \\

\end{longtable}

Point-biserial correlation was conducted on paired continuous and
dichotomous variables, to see if any relationships exist between
extracted concentrations and other variables. The strongest
point-biserial (Pearson) correlation correlations were a near-perfect
positive correlation between cotinine and nicotine (0.98), and moderate
correlations between theophylline and nicotine (0.43), caffeine and
theophylline (0.51) (Table~\ref{tbl-pearson}).

Polychoric correlation was conducted on the dichotomised compounds and
pathological conditions, as well as the discretised dental diseases.
Salicylic acid was removed due to its ubiquitous presence in the sample,
and is likely to cause spurious correlations. Strong correlations were
found between cotinine and nicotine (0.85). Moderate correlations were
found between OA and DDD (0.47), VOP and periodontitis (0.49), SN and
cotinine (0.56), DDD and calculus (-0.42), CMS and caffeine (0.53),
caries and periodontitis (0.52), periodontitis and VOP (0.49),
periodontitis and age-at-death (0.41), nicotine and SN (0.53), calculus
and DDD (-0.42), age-at-death and theophylline (-0.45), theophylline and
age-at-death (-0.45), caffeine and periodontitis (0.49), cotinine and
CMS (0.43). Remaining correlations were weak or absent
(Figure~\ref{fig-polycorr}). Correlations with age will be depressed
because age was largely controlled for in the sample selection.

\begin{figure}

\centering{

\includegraphics{../figures/fig-polycorr-1.pdf}

}

\caption{\label{fig-polycorr}Plot of the polychoric correlations
(\emph{rho}). Larger circles and increased opacity indicates a stronger
correlation coefficient. OA = osteoarthritis; VOP = vertebral
osteophytosis; SN = Schmorl's nodes; DDD = degenerative disc disease; CO
= cribra orbitalia; CMS = chronic maxillary sinusitis; SA = salicylic
acid.}

\end{figure}%

\section{Discussion}\label{discussion}

In this study we were able to extract and identify multiple alkaloids
and salicylic acid from the dental calculus of individuals from
Middenbeemster, a 19th century Dutch archaeological site. We applied
ultra-high-performance liquid chromatography-tandem mass spectrometry
(UHPLC-MS/MS) using a method that was validated by co-occurrence of
drugs and metabolites in dental calculus and blood
(\citeproc{ref-sorensenDrugsCalculus2021}{Sørensen et al., 2021}). Here
we have shown that the method can also be successfully applied to
archaeological dental calculus. We extend findings from previous studies
on alkaloids in archaeological samples by detecting multiple different
alkaloids in dental calculus, including nicotine, cotinine, caffeine,
theophylline, and salicylic acid. The detection of these compounds was
solidified in a replication analysis on different samples from the same
individuals. Cocaine and multiple cannabinoids were also detected during
the first analysis, but were not replicated. We contextualize these
findings within the historical and archaeological evidence for
consumption of these drugs and dietary compounds.

Nicotine and its principal/main metabolite, cotinine, were strongly
positively correlated, both in concentration and presence/absence in
individuals (Table~\ref{tbl-pearson} and Figure~\ref{fig-polycorr}). The
detection of nicotine and cotinine is not surprising, as pipe-smoking in
the Beemsterpolder is well-documented in the literature
(\citeproc{ref-aten400Jaar2012}{Aten et al., 2012};
\citeproc{ref-boumanBegravenis2017}{Bouman, 2017}), and visible on the
skeletal remains as pipe notches
(\citeproc{ref-lemmersMiddenbeemster2013}{Lemmers et al., 2013}). There
is also documented medicinal use of nicotine in the Beemsterpolder,
where a tobacco-smoke enema was used for headaches, respiratory
problems, colds, and drowsiness from around 1780 to 1830
(\citeproc{ref-aten400Jaar2012}{Aten et al., 2012}). In our sample, we
also detected nicotine and cotinine (replicated) in an old adult,
probable female individual. In this particular case it is unlikely that
the compounds entered the dental calculus through pipe-smoking, as the
individual had no visible pipe notches; more likely the tobacco entered
through an alternate mode of consumption, secondhand smoke, or the
aforementioned tobacco-smoke enema.

Theophylline and caffeine were positively correlated in our samples,
though to a lesser extent than nicotine and cotinine, so we are unable
to determine if they originated from the same source
(Table~\ref{tbl-pearson} and Figure~\ref{fig-polycorr}). Caffeine and
theophylline have very similar chemical structures, so we expect they
would experience similar rates of incorporation and degradation,
allowing us to interpret the ratio and correlations between the
compounds. Caffeine is present in coffee, tea, and cocoa beans, with
concentrations slightly higher in coffee
(\citeproc{ref-bispoSimultaneousDetermination2002}{Bispo et al., 2002};
\citeproc{ref-chinCaffeineContent2008}{Chin et al., 2008};
\citeproc{ref-srdjenovicSimultaneousHPLC2008}{Srdjenovic et al., 2008};
\citeproc{ref-stavricVariabilityCaffeine1988}{Stavric et al., 1988}).
Theophylline is present in both coffee beans and tea leaves, but in
negligible quantities
(\citeproc{ref-stavricVariabilityCaffeine1988}{Stavric et al., 1988}).
It is also a primary metabolite of caffeine produced by the liver. Given
the low correlation, there are likely multiple sources of caffeine and
theophylline in the population, with tea and coffee being the most
obvious.\\
Tea consumption had become widespread in the Netherlands by 1820,
reaching all parts of society
(\citeproc{ref-nierstraszTeaTrade2015}{Nierstrasz, 2015, p. 91}).
Historically, we also know that both tea and coffee were consumed in the
Beemsterpolder during the 19th century. `Theegasten' (teatime) was a
special occasion occurring from 15.00-20.00 hours, where tea was served
along with the evening bread
(\citeproc{ref-schuijtemakerTeTheegasten2011}{Schuijtemaker, 2011}).
Many households also owned at least one coffee pot and tea pot
(\citeproc{ref-boumanBegravenis2017}{Bouman, 2017}). Distinguishing
between tea, coffee, and chocolate may be possible by also including
theobromine and comparing ratios of the compounds, as theobromine is
present in higher quantities in chocolate compared to caffeine and
theophylline (\citeproc{ref-alanonAssessmentFlavanol2016}{Alañón et al.,
2016}; \citeproc{ref-bispoSimultaneousDetermination2002}{Bispo et al.,
2002}; \citeproc{ref-stavricVariabilityCaffeine1988}{Stavric et al.,
1988}). However, In addition to oral factors affecting alkaloid uptake
in dental calculus, there is some indication that theobromine does not
preserve well in the archaeological record
(\citeproc{ref-velskoDentalCalculus2017}{Velsko et al., 2017}), and
frequent consumption of all three items would be difficult to parse.
Additionally, we do not understand well enough the effect of the burial
on these specific compounds, and the original concentration of the
compounds in plants can be quite variable
(\citeproc{ref-kingCautionaryTales2017}{King et al., 2017}).

Salicylic acid was found in all but one individual in our sample. It can
be extracted from the bark of willow trees, \emph{Salix alba}, and has
long been used for its pain-relieving properties
(\citeproc{ref-bruinsmaBijdragenTot1872}{Bruinsma, 1872, p. 119}). It is
also present in many plant-based foods
(\citeproc{ref-duthieNaturalSalicylates2011}{Duthie \& Wood, 2011};
\citeproc{ref-malakarNaturallyOccurring2017}{Malakar et al., 2017}),
including potatoes, which were a staple of the Beemsterpolder diet
(\citeproc{ref-aten400Jaar2012}{Aten et al., 2012}). The extracted
quantity from our samples decreased over the three washes, followed by a
sharp increase in the final calculus extraction, which is what we would
expect to see if the salicylic acid was incorporated during life
(Figure~\ref{fig-auth-plot-batch2}). It is important to note that,
especially with salicylic acid, there is a possibility for the compound
to enter the calculus through contact with the surrounding soil.
Salicylic acid is a very mobile organic acid
(\citeproc{ref-badriRegulationFunction2009}{Badri \& Vivanco, 2009};
\citeproc{ref-chenCa2Dependent2001}{Chen et al., 2001}) and the
ubiqutous presence in our samples may be explained by the compound
leching into the dental calculus from the burial environment. We can
therefore not confidently rule out environmental contamination without
analysing samples from the surrounding soil.

Cannabinoids---specifically THC, THCA-A, THCVA, CBD, CBN---were found in
the first batch, but none were replicated in the second batch. Medicinal
use of cannabinoids has been well-established in Europe since
Medieval-times, and it was also grown in the Netherlands
(\citeproc{ref-bruinsmaBijdragenTot1872}{Bruinsma, 1872}).
Administration was most common in the form of concoctions containing
various portions of the cannabis plant for ingestion; not until the late
19th century did it become recommended to smoke it for more immediate
effects (\citeproc{ref-clarkeCannabisEvolution2013}{Clarke, 2013}).
Dutch medicinal preparations were used to treat a variety of ailments
and symptoms, including pain, inflammation, various stomach ailments,
gout, and joint pains
(\citeproc{ref-clarkeCannabisEvolution2013}{Clarke, 2013}). Because
cannabinoids have an affinity for protein-binding, they are less likely
to diffuse from serum to saliva
(\citeproc{ref-coneInterpretationOral2007}{Cone \& Huestis, 2007}). This
may make them difficult to detect in dental calculus unless the
cannibinoids were consumed orally; even then, the overall instability of
some cannabinoids could also limit detection
(\citeproc{ref-lindholstLongTerm2010}{Lindholst, 2010};
\citeproc{ref-sorensenEffectAntioxidants2018}{Sørensen \& Hasselstrøm,
2018}). Given the lack of replication, we cannot with security confirm
that cannabis was used by the Beemster population.

Despite many of our sampled individuals having lived during the height
of the opium era in the Netherlands
(\citeproc{ref-machtHistoryOpium1915}{Macht, 1915}), none of the
targeted opioids (morphine, codeine, thebaine, papaverine, norcodeine,
noscapine) were detected. The absence of opioids could be a result of
the people ascribing more to the ``traditional'' rather than
``scientific'' medicine, although laudanum and another opium containing
concoction was part of the ``traditional'' medicine in the Netherlands
(\citeproc{ref-leuwProhibitionLegalization1994}{Leuw \& Marshall,
1994}), including Middenbeemster (\citeproc{ref-aten400Jaar2012}{Aten et
al., 2012}). It was also generally considered a drug of the upper class
(\citeproc{ref-scheltemaOpiumTrade1907}{Scheltema, 1907}), and may have
been more common in urban centers. The absence could also be attributed
to postmortem degradation. It has been shown that, while morphine is
abundant in opium, it degrades rapidly. Thebaine and papaverine are more
resistant to various ageing processes
(\citeproc{ref-chovanecOpiumMasses2012}{Chovanec et al., 2012}),
however, these were also absent from our samples.

The only strictly modern compound (at least in a European context)
detected in the sample was cocaine, which was detected in the first
batch of samples. Our sample is derived from an early--mid 19th century
population, and cocaine was isolated in 1860 by Albert Niemann, and
entered popular medical practice in 1884. Coca arrived in Europe as
early as 1771, but as botanical specimens rather than for consumption,
and there were also issues importing enough viable specimens of coca for
cocaine extraction (\citeproc{ref-abucaCocaTrade2019}{Abduca, 2019, p.
108}; \citeproc{ref-mortimerHistoryCoca1901}{Mortimer, 1901, p. 179}).
This would have been the first case of coca-leaf-consumption in Europe;
however, we were unable to replicate any of the cocaine results in the
second batch. We suspect that the original detection of cocaine was a
result of lab contamination during analysis.

We explored the relationship between detected compounds and various
skeletal indicators, such as pathological and dental lesions,
preservation, and pipe notches. We found some evidence to suggest that
preservation of the skeleton influences the recovery of compounds from
the dental calculus, with well-preserved skeletons potentially serving
as a better target for sampling.\\
We found a positive correlation between CMS and nicotine, which may be
indicative of the impact tobacco smoking had on the respiratory health
of the Beemster inhabitants. Tobacco smoke may play a significant role
in diseases of the upper respiratory tract, including chronic maxillary
sinusitis (\citeproc{ref-rehImpactTobacco2012}{Reh et al., 2012}).
Although the mechanisms by which smoking increases the risk of
infections is not fully understood, solid evidence has been presented
linking tobacco smoke to increased mucosal permeability and impairment
of mucociliary clearance
(\citeproc{ref-arcaviCigaretteSmoking2004}{Arcavi \& Benowitz, 2004}).
Such changes, together with an altered immunologic response, are thought
to predispose to the development of chronic maxillary sinusitis
(\citeproc{ref-slavinDiagnosisManagement2005}{Slavin et al., 2005}).\\
We also observed a moderate positive correlation between chronic
maxillary sinusitis and caffeine which contradicts previous research
linking chronic coffee consumption with a positive effect on the
respiratory system, suggesting a preventive association between caffeine
intake and pneumonia (e.g. \citeproc{ref-alfaroChronicCoffee2018}{Alfaro
et al., 2018}; \citeproc{ref-kondoAssociationCoffee2021}{Kondo et al.,
2021}). However, while the lower respiratory tract seems to benefit from
chronic coffee consumption, it is possible that elevated caffeine intake
impacts mucosal moisture due to its dehydrating effect
(\citeproc{ref-maughanCaffeineIngestion2003}{Maughan \& Griffin, 2003}),
thereby exposing individuals to greater risk of sinus infection.

The detection of nicotine in dental calculus has previously been
presented by Eerkens and colleagues
(\citeproc{ref-eerkensDentalCalculus2018}{2018}) in two individuals from
pre-contact California. They also targeted caffeine, cotinine, and
theophylline in their samples, but were unable to detect any of them. It
remains to be seen whether this is due to differences in methods used,
or due to our samples being more recent. They also suggest that the
choice of tooth for sampling may impact the detection of certain
compounds, as the incorporation in dental calculus may depend on the
mode of consumption. Tobacco smokers may have more nicotine present in
calculus on incisors, whereas tobacco chewers may have more on molars
(\citeproc{ref-eerkensDentalCalculus2018}{Eerkens et al., 2018}).
However, sampling may not be limited to mode of consumption. The
presence of cotinine suggests that the excretion of a compound after
being metabolised in the body is also a source of deposition, and that
deposition of alkaloids in dental calculus can occur both on the way
into the body, i.e.~during consumption, and on the way out,
i.e.~disposal of waste products via saliva secretion into the mouth.
Especially mucin-rich saliva from the sublingual and submandibular
glands preferentially binds toxins
(\citeproc{ref-doddsHealthBenefits2005}{Dodds et al., 2005}), and since
these glands are located closest to the lower incisors, they may be the
most effective target for these studies. This has yet to be
systematically tested in archaeological dental calculus. Because we
homogenised samples from multiple teeth of an individual, we were unable
to test the effect of oral biogeography. It is also possible that
resident microflora within biofilms contribute to alkaloid breakdown and
that the presence of caffeine and nicotine metabolites following direct
ingestion can be explained by this pathway. However, the literature on
biofilm biodegradation of alkaloids is limited, and \emph{in vitro}
studies have only found minimal contributions by certain oral bacteria
in isolation (\citeproc{ref-cogoVitroEvaluation2008}{Cogo et al., 2008};
\citeproc{ref-sunMetabolomicsEvaluation2016}{Sun et al., 2016}); it is
possible that a larger role is played by oral bacteria within larger,
more metabolically active communities, e.g.~biofilms
(\citeproc{ref-takahashiOralMicrobiome2015}{Takahashi, 2015}).

Targeting individuals with moderate-to-large calculus deposits likely
biased our sample, as the presence of calculus may increase the risk of
premature death (\citeproc{ref-yaussyCalculusSurvivorship2019}{Yaussy \&
DeWitte, 2019}). Additionally, periodontal disease (often linked to the
presence of calculus) is a risk-factor for respiratory diseases, if
periodontal and respiratory pathogens enter the bloodstream
(\citeproc{ref-azarpazhoohSystematicReview2006}{Azarpazhooh \& Leake,
2006}; \citeproc{ref-scannapiecoRoleOral1999}{Scannapieco, 1999};
\citeproc{ref-scannapiecoPotentialAssociations2001}{Scannapieco \& Ho,
2001}). In our sample, the percentage of chronic maxillary sinusitis
(37.0\%) is lower than in another (more representative) male sample
(44.1\%) (\citeproc{ref-casnaUrbanizationRespiratory2021}{Casna et al.,
2021}), and the caries percentage is similarly lower in our sample
(17.6\%) than a more representative sample (22.9\%)
(\citeproc{ref-lemmersMiddenbeemster2013}{Lemmers et al., 2013}).\\
We used the presence/absence of a pipe notch and concurrent detection of
tobacco as a crude estimate of the accuracy of the method, which we
found to be around 59.3\%. This is a very rough estimate, as the
presence of a pipe notch is likely not a perfect indicator of whether or
not someone consumed tobacco. Dental calculus is also more transient
than for example bone, as it can become dislodged during life,
intentionally or unintentionally, eliminating all trace of the alkaloids
consumed prior to its removal.\\
Following burial, compound stability over time will play a large role,
as will microbial degradation of compounds by bacteria and fungi in soil
(\citeproc{ref-liuNicotinedegradingMicroorganisms2015}{Liu et al.,
2015}), as well as the soil environment, such as temperature, pH, and
oxygen availability (\citeproc{ref-lindholstLongTerm2010}{Lindholst,
2010}; \citeproc{ref-mackiePreservationMetaproteome2017}{Mackie et al.,
2017}).\\
Due to this, quantitation of the detected compounds may have limited
value in archaeological samples due to degradation, and will greatly
affect our correlations related to concentration. The detected quantity
of a compound will also depend on the quantity in dental calculus during
life, which is largely controlled by the quantity consumed, how often
the calculus was disrupted/removed, metabolic breakdown of the compound,
and inter- and intra-individual factors related to stages of biofilm
maturation (\citeproc{ref-lustmannScanningElectron1976}{Lustmann et al.,
1976}; \citeproc{ref-velskoMicrobialDifferences2019}{Velsko et al.,
2019}; \citeproc{ref-zijngeBiofilmArchitecture2010}{Zijnge et al.,
2010}). In short, this means it is not really possible to equate the
absence of a compound as evidence for the absence of consumption, which
complicates the interpretation of our results. We have attempted to
minimise errors occurring due to this limitation by including a
relatively large sample of individuals and replicating our analysis.
Although, given the relatively low detection rate seen in tobacco, this
remains a major limitation, and will likely be compounded by increasing
antiquity of the samples.

Future studies should explore how sampling from various types of teeth
and their position in the mouth affects the probability of a compound
becoming entrapped in dental calculus. This may also be related to
properties within the oral cavity, as well as chemical properties of the
compounds, which facilitate or reduce the incorporation-potential, and
which incorporation pathways are more likely for a given compound.\\
We only targeted drugs that were included in the forensic toxicological
screenings, and therefore only covered a limited number of the potential
compounds that could be of interest for exploring past diets and
medicinal treatments. The list of targeted compounds can be expanded as
we discover more potential targets based on which specific
compounds/metabolites are more likely to be incorporated and preserved
in dental calculus.\\
There is an increasing interest in using oral fluid as a means of
detecting alkaloids in living individuals due to the non-invasive nature
of the testing compared to blood and urine sampling
(\citeproc{ref-coneSalivaTesting1993}{Cone, 1993};
\citeproc{ref-valenDetermination212017}{Valen et al., 2017}). These
\emph{in vivo} studies are a valuable source of method validation and
can help determine the feasibility of detecting certain alkaloids in
oral fluid and, subsequently, dental calculus. Archaeologists, though,
will likely be responsible for exploring dental-calculus-specific
incorporation and retention of alkaloids, as well as their long-term
preservation in the burial environment. Finally, following our
experience with salicylic acid, we encourage all future studies to
ensure that a control sample is taken from the soil, either from the
soil surrounding the individual, or, ideally, directly from the skeletal
remains. This should preferably happen before cleaning, but there will
often be soil left over in cavities (e.g.~nasal cavity, orbit, auditory
meatus).

While a major limitation is the uncertainty surrounding whether or not a
compound is actually absent, the power of the method lies in the ability
to detect dietary and other compounds that were incorporated via
multiple consumption pathways that are not detected by other methods.
Taking tobacco consumption as an example; while pipe notches are a
useful way to identify tobacco consumption, pipe smoking was not the
only mode of tobacco consumption, with others including chewing,
drinking, cigars, and snuff
(\citeproc{ref-goodmanTobaccoHistory1994}{Goodman, 1994, p. 67}).
Pipe-smoking was mainly practised by males
(\citeproc{ref-eerkensDentalCalculus2018}{Eerkens et al., 2018};
\citeproc{ref-lemmersMiddenbeemster2013}{Lemmers et al., 2013}), so
methods like the one presented here are suitable for exploring tobacco
consumption in an entire society, rather than a trivial subset of past
populations. Combined with other methods, it can also give us a more
complete picture of dietary patterns and medicinal/recreational
plant-use in the past by capturing multiple possible incorporation
pathways of dietary (and other) compounds.

\section{Conclusions}\label{conclusions}

This preliminary study outlines the benefits of using calculus to target
a variety of compounds that could have been consumed as medicine or
diet. This method allows us to directly address specific individuals,
which can be especially useful in individuals that are not always
well-documented in historic documentation, such as rural communities,
children and women. We also show that there are many limitations that
will need to be addressed going forward with this type of analysis, and
stress the need for more systematic research on the consumption of
alkaloid-containing items and their subsequent concentration and
preservation in dental calculus, in addition to how mode of consumption
may affect concentrations on different parts of the dentition. Another
limitation of dental calculus as a medium is the inter- and
intra-individual variability of its formation and the many factors that
can influence incorporation and retention of molecules and particles;
however, in the absence of hair and serum (quite uncommon in
archaeology), dental calculus represents an impressive long-term
reservoir of information regarding the consumption of various alkaloids,
whether dietary, medicinal, recreational, or otherwise.

\section*{Acknowledgements}\label{acknowledgements}
\addcontentsline{toc}{section}{Acknowledgements}

We wish to thank Kirsten Ziesemer for helping track down the calculus
samples from her studies. Additionally, we owe special thanks to Vincent
Falger and Kees de Groot from the Middenbeemster Historical Society for
their input on early draft manuscripts and a lovely guided tour of
Middenbeemster. We also thank Louise Le Meillour for taking on our
preprint and providing helpful comments, and Mario Zimmerman and two
other anonymous reviewers for their time and invaluable insights.

\section*{Funding}\label{funding}
\addcontentsline{toc}{section}{Funding}

This research has received funding from the European Research Council
under the European Union's Horizon 2020 research and innovation program,
grant agreement number STG--677576 (``HARVEST'').

\section*{Conflict of interest
disclosure}\label{conflict-of-interest-disclosure}
\addcontentsline{toc}{section}{Conflict of interest disclosure}

The authors have no conflicts to declare.

\section*{Data Availability
Statement}\label{data-availability-statement}
\addcontentsline{toc}{section}{Data Availability Statement}

All raw data is available on Zenodo
(\url{https://doi.org/10.5281/zenodo.8061483}). Analysis scripts, and
the source code for the manuscript and supplementary materials are
available as a research compendium
(\url{https://doi.org/10.5281/zenodo.7649824}) using the structure
recommended by the \textbf{rrtools} R package
(\citeproc{ref-Rrrtools}{Marwick, 2019}).

\section*{References}\label{references}
\addcontentsline{toc}{section}{References}

\phantomsection\label{refs}
\begin{CSLReferences}{1}{0}
\bibitem[\citeproctext]{ref-abucaCocaTrade2019}
Abduca, R. (2019). Coca leaf transfers to {Europe}. {Effects} on the
consumption of coca in {North-western Argentina}. In M. Kaller \& F.
Jacob (Eds.), \emph{Transatlantic {Trade} and {Global Cultural Transfers
Since} 1492: {More} than {Commodities}}. {Routledge}.
\url{https://books.google.com?id=13imDwAAQBAJ}

\bibitem[\citeproctext]{ref-alanonAssessmentFlavanol2016}
Alañón, M. E., Castle, S. M., Siswanto, P. J., Cifuentes-Gómez, T., \&
Spencer, J. P. E. (2016). Assessment of flavanol stereoisomers and
caffeine and theobromine content in commercial chocolates. \emph{Food
Chemistry}, \emph{208}, 177--184.
\url{https://doi.org/10.1016/j.foodchem.2016.03.116}

\bibitem[\citeproctext]{ref-alfaroChronicCoffee2018}
Alfaro, T. M., Monteiro, R. A., Cunha, R. A., \& Cordeiro, C. R. (2018).
Chronic coffee consumption and respiratory disease: {A} systematic
review. \emph{The Clinical Respiratory Journal}, \emph{12}(3),
1283--1294. \url{https://doi.org/10.1111/crj.12662}

\bibitem[\citeproctext]{ref-arcaviCigaretteSmoking2004}
Arcavi, L., \& Benowitz, N. L. (2004). Cigarette {Smoking} and
{Infection}. \emph{Archives of Internal Medicine}, \emph{164}(20),
2206--2216. \url{https://doi.org/10.1001/archinte.164.20.2206}

\bibitem[\citeproctext]{ref-aten400Jaar2012}
Aten, D., Bossaers, K. W. J. M., \& Misset, C. (2012). \emph{400 jaar
Beemster: 1612-2012}. {Stichting Uitgeverij Noord-Holland}.

\bibitem[\citeproctext]{ref-azarpazhoohSystematicReview2006}
Azarpazhooh, A., \& Leake, J. L. (2006). Systematic {Review} of the
{Association Between Respiratory Diseases} and {Oral Health}.
\emph{Journal of Periodontology}, \emph{77}(9), 1465--1482.
\url{https://doi.org/10.1902/jop.2006.060010}

\bibitem[\citeproctext]{ref-badriRegulationFunction2009}
Badri, D. V., \& Vivanco, J. M. (2009). Regulation and function of root
exudates. \emph{Plant, Cell \& Environment}, \emph{32}(6), 666--681.
\url{https://doi.org/10.1111/j.1365-3040.2009.01926.x}

\bibitem[\citeproctext]{ref-bispoSimultaneousDetermination2002}
Bispo, M. S., Veloso, M. C. C., Pinheiro, H. L. C., De Oliveira, R. F.
S., Reis, J. O. N., \& De Andrade, J. B. (2002). Simultaneous
{Determination} of {Caffeine}, {Theobromine}, and {Theophylline} by
{High-Performance Liquid Chromatography}. \emph{Journal of
Chromatographic Science}, \emph{40}(1), 45--48.
\url{https://doi.org/10.1093/chromsci/40.1.45}

\bibitem[\citeproctext]{ref-boocockMaxillarySinusitis1995}
Boocock, P., Roberts, C. A., \& Manchester, K. (1995). Maxillary
sinusitis in {Medieval Chichester}, {England}. \emph{American Journal of
Physical Anthropology}, \emph{98}(4), 483--495.
\url{https://doi.org/10.1002/ajpa.1330980408}

\bibitem[\citeproctext]{ref-boumanBegravenis2017}
Bouman, J. (2017). De Begravenis. \emph{De Nieuwe Schouwschuit},
\emph{15}, 11--15.
\url{https://www.historischgenootschapbeemster.nl/wp-content/uploads/De_Nieuwe_Schouwschuit_15e_jaargang_november_2017.pdf}

\bibitem[\citeproctext]{ref-SucheyBrooks1990}
Brooks, S., \& Suchey, J. M. (1990). Skeletal age determination based on
the os pubis: {A} comparison of the {Acsádi-Nemeskéri} and
{Suchey-Brooks} methods. \emph{Human Evolution}, \emph{5}(3), 227--238.
\url{https://doi.org/10.1007/BF02437238}

\bibitem[\citeproctext]{ref-brothwellDiggingBones1981}
Brothwell, D. (1981). \emph{Digging up {Bones}: {The} excavation,
treatment and study of human skeletal remains} (3rd ed.). {British
Museum (Natural History)}.

\bibitem[\citeproctext]{ref-bruinsmaBijdragenTot1872}
Bruinsma, J. J. (1872). \emph{Bijdragen tot de {Geneeskundige
Plaatsbeschrijving} van {Nederland}}. {Van Weelden en Mingelen}.
\url{https://dlcs.io/pdf/wellcome/pdf-item/b24874140/0}

\bibitem[\citeproctext]{ref-buckberryAuricular2002}
Buckberry, J. L., \& Chamberlain, A. T. (2002). Age estimation from the
auricular surface of the ilium: A revised method. \emph{American Journal
of Physical Anthropology}, \emph{119}(3), 231--239.
\url{https://doi.org/10.1002/ajpa.10130}

\bibitem[\citeproctext]{ref-buckleyDentalCalculus2014}
Buckley, S., Usai, D., Jakob, T., Radini, A., \& Hardy, K. (2014).
Dental {Calculus Reveals Unique Insights} into {Food Items}, {Cooking}
and {Plant Processing} in {Prehistoric Central Sudan}. \emph{PLOS ONE},
\emph{9}(7), e100808. \url{https://doi.org/10.1371/journal.pone.0100808}

\bibitem[\citeproctext]{ref-Standards1994}
Buikstra, J. E., \& Ubelaker, D. H. (1994). Standards for data
collection from human skeletal remains: {Proceedings} of a seminar at
the {Field Museum} of {Natural History} ({Arkansas Archaeology Research
Series} 44). \emph{Fayetteville Arkansas Archaeological Survey}.

\bibitem[\citeproctext]{ref-casnaUrbanizationRespiratory2021}
Casna, M., Burrell, C. L., Schats, R., Hoogland, M. L. P., \& Schrader,
S. A. (2021). Urbanization and respiratory stress in the {Northern Low
Countries}: {A} comparative study of chronic maxillary sinusitis in two
early modern sites from the {Netherlands} ({AD} 1626--1866).
\emph{International Journal of Osteoarchaeology}, \emph{31}(5),
891--901. \url{https://doi.org/10.1002/oa.3006}

\bibitem[\citeproctext]{ref-chenCa2Dependent2001}
Chen, H., Hou, W., Kuć, J., \& Lin, Y. (2001). Ca2+‐dependent and
{Ca2}+‐independent excretion modes of salicylic acid in tobacco cell
suspension culture. \emph{Journal of Experimental Botany},
\emph{52}(359), 1219--1226.
\url{https://doi.org/10.1093/jexbot/52.359.1219}

\bibitem[\citeproctext]{ref-chinCaffeineContent2008}
Chin, J. M., Merves, M. L., Goldberger, B. A., Sampson-Cone, A., \&
Cone, E. J. (2008). Caffeine {Content} of {Brewed Teas}. \emph{Journal
of Analytical Toxicology}, \emph{32}(8), 702--704.
\url{https://doi.org/10.1093/jat/32.8.702}

\bibitem[\citeproctext]{ref-chovanecOpiumMasses2012}
Chovanec, Z., Rafferty, S., \& Swiny, S. (2012). Opium for the {Masses}.
\emph{Ethnoarchaeology}, \emph{4}(1), 5--36.
\url{https://doi.org/10.1179/eth.2012.4.1.5}

\bibitem[\citeproctext]{ref-clarkeCannabisEvolution2013}
Clarke, R. (2013). \emph{Cannabis : {Evolution} and {Ethnobotany}}.
{University of California Press}.

\bibitem[\citeproctext]{ref-cogoVitroEvaluation2008}
Cogo, K., Montan, M. F., Bergamaschi, C. de C., D. Andrade, E., Rosalen,
P. L., \& Groppo, F. C. (2008). In vitro evaluation of the effect of
nicotine, cotinine, and caffeine on oral microorganisms. \emph{Canadian
Journal of Microbiology}, \emph{54}(6), 501--508.
\url{https://doi.org/10.1139/W08-032}

\bibitem[\citeproctext]{ref-coneSalivaTesting1993}
Cone, E. J. (1993). Saliva {Testing} for {Drugs} of {Abuse}.
\emph{Annals of the New York Academy of Sciences}, \emph{694}(1),
91--127. \url{https://doi.org/10.1111/j.1749-6632.1993.tb18346.x}

\bibitem[\citeproctext]{ref-coneInterpretationOral2007}
Cone, E. J., \& Huestis, M. A. (2007). Interpretation of {Oral Fluid
Tests} for {Drugs} of {Abuse}. \emph{Annals of the New York Academy of
Sciences}, \emph{1098}, 51--103.
\url{https://doi.org/10.1196/annals.1384.037}

\bibitem[\citeproctext]{ref-doddsHealthBenefits2005}
Dodds, M. W. J., Johnson, D. A., \& Yeh, C.-K. (2005). Health benefits
of saliva: A review. \emph{Journal of Dentistry}, \emph{33}(3),
223--233. \url{https://doi.org/10.1016/j.jdent.2004.10.009}

\bibitem[\citeproctext]{ref-duthieNaturalSalicylates2011}
Duthie, G. G., \& Wood, A. D. (2011). Natural salicylates: Foods ,
functions and disease prevention. \emph{Food \& Function}, \emph{2}(9),
515--520. \url{https://doi.org/10.1039/C1FO10128E}

\bibitem[\citeproctext]{ref-echeverriaNicotineHair2013}
Echeverría, J., \& Niemeyer, H. M. (2013). Nicotine in the hair of
mummies from {San Pedro} de {Atacama} ({Northern Chile}). \emph{Journal
of Archaeological Science}, \emph{40}(10), 3561--3568.
\url{https://doi.org/10.1016/j.jas.2013.04.030}

\bibitem[\citeproctext]{ref-eerkensDentalCalculus2018}
Eerkens, J. W., Tushingham, S., Brownstein, K. J., Garibay, R., Perez,
K., Murga, E., Kaijankoski, P., Rosenthal, J. S., \& Gang, D. R. (2018).
Dental calculus as a source of ancient alkaloids: {Detection} of
nicotine by {LC-MS} in calculus samples from the {Americas}.
\emph{Journal of Archaeological Science: Reports}, \emph{18}, 509--515.
\url{https://doi.org/10.1016/j.jasrep.2018.02.004}

\bibitem[\citeproctext]{ref-gismondiMultidisciplinaryApproach2020}
Gismondi, A., Baldoni, M., Gnes, M., Scorrano, G., D'Agostino, A.,
Marco, G. D., Calabria, G., Petrucci, M., Müldner, G., Tersch, M. V.,
Nardi, A., Enei, F., Canini, A., Rickards, O., Alexander, M., \&
Martínez-Labarga, C. (2020). A multidisciplinary approach for
investigating dietary and medicinal habits of the {Medieval} population
of {Santa Severa} (7th-15th centuries, {Rome}, {Italy}). \emph{PLOS
ONE}, \emph{15}(1), e0227433.
\url{https://doi.org/10.1371/journal.pone.0227433}

\bibitem[\citeproctext]{ref-goodmanTobaccoHistory1994}
Goodman, J. (1994). \emph{Tobacco in history: The cultures of
dependence}. {Routledge}.

\bibitem[\citeproctext]{ref-greeneQuantifyingCalculus2005}
Greene, T. R., Kuba, C. L., \& Irish, J. D. (2005). Quantifying
calculus: {A} suggested new approach for recording an important
indicator of diet and dental health. \emph{HOMO - Journal of Comparative
Human Biology}, \emph{56}(2), 119--132.
\url{https://doi.org/10.1016/j.jchb.2005.02.002}

\bibitem[\citeproctext]{ref-huangDecipheringGenetic2023}
Huang, Y., Shan, Y., Zhang, W., Lee, A. M., Li, F., Stranger, B. E., \&
Huang, R. S. (2023). Deciphering genetic causes for sex differences in
human health through drug metabolism and transporter genes. \emph{Nature
Communications}, \emph{14}(1, 1), 175.
\url{https://doi.org/10.1038/s41467-023-35808-6}

\bibitem[\citeproctext]{ref-jinSupragingivalCalculus2002}
Jin, Y., \& Yip, H.-K. (2002). Supragingival {Calculus}: {Formation} and
{Control}. \emph{Critical Reviews in Oral Biology \& Medicine}.
\url{https://doi.org/10.1177/154411130201300506}

\bibitem[\citeproctext]{ref-kingCautionaryTales2017}
King, A., Powis, T. G., Cheong, K. F., \& Gaikwad, N. W. (2017).
Cautionary tales on the identification of caffeinated beverages in
{North America}. \emph{Journal of Archaeological Science}, \emph{85},
30--40. \url{https://doi.org/10.1016/j.jas.2017.06.006}

\bibitem[\citeproctext]{ref-kondoAssociationCoffee2021}
Kondo, K., Suzuki, K., Washio, M., Ohfuji, S., Adachi, S., Kan, S.,
Imai, S., Yoshimura, K., Miyashita, N., Fujisawa, N., Maeda, A.,
Fukushima, W., \& Hirota, Y. (2021). Association between coffee and
green tea intake and pneumonia among the {Japanese} elderly: A
case-control study. \emph{Scientific Reports}, \emph{11}(1, 1), 5570.
\url{https://doi.org/10.1038/s41598-021-84348-w}

\bibitem[\citeproctext]{ref-lemmersMiddenbeemster2013}
Lemmers, S. A. M., Schats, R., Hoogland, M. L. P., \& Waters-Rist, A.
(2013). Fysisch antropologische analyse Middenbeemster. In \emph{De
begravingen bij de Keyserkerk te Middenbeemster} (pp. 35--60).

\bibitem[\citeproctext]{ref-leuwProhibitionLegalization1994}
Leuw, E., \& Marshall, I. H. (1994). \emph{Between {Prohibition} and
{Legalization}: {The Dutch Experiment} in {Drug Policy}}. {Kugler
Publications}. \url{https://books.google.com?id=2mAVkStNG5EC}

\bibitem[\citeproctext]{ref-lindholstLongTerm2010}
Lindholst, C. (2010). Long term stability of cannabis resin and cannabis
extracts. \emph{Australian Journal of Forensic Sciences}, \emph{42}(3),
181--190. \url{https://doi.org/10.1080/00450610903258144}

\bibitem[\citeproctext]{ref-liuNicotinedegradingMicroorganisms2015}
Liu, J., Ma, G., Chen, T., Hou, Y., Yang, S., Zhang, K.-Q., \& Yang, J.
(2015). Nicotine-degrading microorganisms and their potential
applications. \emph{Applied Microbiology and Biotechnology},
\emph{99}(9), 3775--3785.
\url{https://doi.org/10.1007/s00253-015-6525-1}

\bibitem[\citeproctext]{ref-lovejoyAuricular1985}
Lovejoy, C. O., Meindl, R. S., Pryzbeck, T. R., \& Mensforth, R. P.
(1985). Chronological metamorphosis of the auricular surface of the
ilium: {A} new method for the determination of adult skeletal age at
death. \emph{American Journal of Physical Anthropology}, \emph{68}(1),
15--28. \url{https://doi.org/10.1002/ajpa.1330680103}

\bibitem[\citeproctext]{ref-lustmannScanningElectron1976}
Lustmann, J., Lewin-Epstein, J., \& Shteyer, A. (1976). Scanning
electron microscopy of dental calculus. \emph{Calcified Tissue
Research}, \emph{21}(1), 47--55.
\url{https://doi.org/10.1007/BF02547382}

\bibitem[\citeproctext]{ref-maatManualPhysical2005}
Maat, G., \& Mastwijk, R. (2005). Manual for the physical
anthropological report. \emph{Barge's Anthropologica}, \emph{6}.

\bibitem[\citeproctext]{ref-machtHistoryOpium1915}
Macht, D. I. (1915). The history of opium and some of its preparations
and alkaloids. \emph{The Journal of the American Medical Association},
\emph{LXIV}(6), 5.

\bibitem[\citeproctext]{ref-mackiePreservationMetaproteome2017}
Mackie, M., Hendy, J., Lowe, A. D., Sperduti, A., Holst, M., Collins, M.
J., \& Speller, C. F. (2017). Preservation of the metaproteome:
Variability of protein preservation in ancient dental calculus.
\emph{STAR: Science \& Technology of Archaeological Research},
\emph{3}(1), 58--70. \url{https://doi.org/10.1080/20548923.2017.1361629}

\bibitem[\citeproctext]{ref-malakarNaturallyOccurring2017}
Malakar, S., Gibson, P. R., Barrett, J. S., \& Muir, J. G. (2017).
Naturally occurring dietary salicylates: {A} closer look at common
{Australian} foods. \emph{Journal of Food Composition and Analysis},
\emph{57}, 31--39. \url{https://doi.org/10.1016/j.jfca.2016.12.008}

\bibitem[\citeproctext]{ref-Rrrtools}
Marwick, B. (2019). \emph{Rrtools: {Creates} a reproducible research
compendium} {[}Manual{]}. \url{https://github.com/benmarwick/rrtools}

\bibitem[\citeproctext]{ref-maughanCaffeineIngestion2003}
Maughan, R. J., \& Griffin, J. (2003). Caffeine ingestion and fluid
balance: A review. \emph{Journal of Human Nutrition and Dietetics},
\emph{16}(6), 411--420.
\url{https://doi.org/10.1046/j.1365-277X.2003.00477.x}

\bibitem[\citeproctext]{ref-meindlSutureClosure1985}
Meindl, R. S., \& Lovejoy, C. O. (1985). Ectocranial suture closure: {A}
revised method for the determination of skeletal age at death based on
the lateral-anterior sutures. \emph{American Journal of Physical
Anthropology}, \emph{68}(1), 57--66.
\url{https://doi.org/10.1002/ajpa.1330680106}

\bibitem[\citeproctext]{ref-milmanOralFluid2011}
Milman, G., Schwope, D. M., Schwilke, E. W., Darwin, W. D., Kelly, D.
L., Goodwin, R. S., Gorelick, D. A., \& Huestis, M. A. (2011). Oral
{Fluid} and {Plasma Cannabinoid Ratios} after {Around-the-Clock
Controlled Oral Δ9-Tetrahydrocannabinol Administration}. \emph{Clinical
Chemistry}, \emph{57}(11), 1597--1606.
\url{https://doi.org/10.1373/clinchem.2011.169490}

\bibitem[\citeproctext]{ref-mortimerHistoryCoca1901}
Mortimer, W. G. (1901). \emph{Peru. {History} of coca, "the divine
plant" of the {Incas}; with an introductory account of the {Incas}, and
of the {Andean Indians} of to-day}. {New York, J. H. Vail \& Company}.
\url{http://archive.org/details/peruhistoryofcoc00mortrich}

\bibitem[\citeproctext]{ref-nierstraszTeaTrade2015}
Nierstrasz, C. (2015). \emph{Rivalry for {Trade} in {Tea} and
{Textiles}: {The English} and {Dutch East India} companies
(1700--1800)}. {Springer}.
\url{https://books.google.com?id=uwtaCwAAQBAJ}

\bibitem[\citeproctext]{ref-ogaldeIdentificationPsychoactive2009}
Ogalde, J. P., Arriaza, B. T., \& Soto, E. C. (2009). Identification of
psychoactive alkaloids in ancient {Andean} human hair by gas
chromatography/mass spectrometry. \emph{Journal of Archaeological
Science}, \emph{36}(2), 467--472.
\url{https://doi.org/10.1016/j.jas.2008.09.036}

\bibitem[\citeproctext]{ref-palmerActivityReconstruction2016}
Palmer, J. L. A., Hoogland, M. H. L., \& Waters‐Rist, A. L. (2016).
Activity {Reconstruction} of {Post}‐{Medieval Dutch Rural Villagers}
from {Upper Limb Osteoarthritis} and {Entheseal Changes}.
\emph{International Journal of Osteoarchaeology}, \emph{26}(1), 78--92.
\url{https://doi.org/10.1002/oa.2397}

\bibitem[\citeproctext]{ref-Rbase}
R Core Team. (2020). \emph{R: {A} language and environment for
statistical computing} {[}Manual{]}. {R Foundation for Statistical
Computing}; {R Foundation for Statistical Computing}.
\url{https://www.R-project.org/}

\bibitem[\citeproctext]{ref-raffertyCurrentResearch2012}
Rafferty, S. M., Lednev, I., Virkler, K., \& Chovanec, Z. (2012).
Current research on smoking pipe residues. \emph{Journal of
Archaeological Science}, \emph{39}(7), 1951--1959.
\url{https://doi.org/10.1016/j.jas.2012.02.001}

\bibitem[\citeproctext]{ref-rehImpactTobacco2012}
Reh, D. D., Higgins, T. S., \& Smith, T. L. (2012). Impact of {Tobacco
Smoke} on {Chronic Rhinosinusitis} -- {A Review} of the {Literature}.
\emph{International Forum of Allergy \& Rhinology}, \emph{2}(5), 362.
\url{https://doi.org/10.1002/alr.21054}

\bibitem[\citeproctext]{ref-Rpsych}
Revelle, W. (2022). \emph{Psych: {Procedures} for psychological,
psychometric, and personality research} {[}Manual{]}. {Northwestern
University}. \url{https://CRAN.R-project.org/package=psych}

\bibitem[\citeproctext]{ref-rogersPalaeopathologyJoint2000}
Rogers, J. (2000). The palaeopathology of joint disease. In M. Cox \& S.
Mays (Eds.), \emph{Human osteology : {In} archaeology and forensic
science.} (1st ed, pp. 163--182). {Cambridge University Press}.
\url{https://login.ezproxy.leidenuniv.nl:2443/login?URL=https://search.ebscohost.com/login.aspx?direct=true&db=e000xww&AN=40641&site=ehost-live}

\bibitem[\citeproctext]{ref-scannapiecoRoleOral1999}
Scannapieco, F. A. (1999). Role of {Oral Bacteria} in {Respiratory
Infection}. \emph{Journal of Periodontology}, \emph{70}(7), 793--802.
\url{https://doi.org/10.1902/jop.1999.70.7.793}

\bibitem[\citeproctext]{ref-scannapiecoPotentialAssociations2001}
Scannapieco, F. A., \& Ho, A. W. (2001). Potential {Associations Between
Chronic Respiratory Disease} and {Periodontal Disease}: {Analysis} of
{National Health} and {Nutrition Examination Survey III}. \emph{Journal
of Periodontology}, \emph{72}(1), 50--56.
\url{https://doi.org/10.1902/jop.2001.72.1.50}

\bibitem[\citeproctext]{ref-scheltemaOpiumTrade1907}
Scheltema, J. F. (1907). The {Opium Trade} in the {Dutch East Indies}.
{I}. \emph{American Journal of Sociology}, \emph{13}(1), 79--112.

\bibitem[\citeproctext]{ref-schuijtemakerTeTheegasten2011}
Schuijtemaker, D. (2011). Te Theegasten. \emph{De Nieuwe Schouwschuit},
\emph{9}, 16--17.

\bibitem[\citeproctext]{ref-slavinDiagnosisManagement2005}
Slavin, R. G., Spector, S. L., Bernstein, I. L., Slavin, R. G., Kaliner,
M. A., Kennedy, D. W., Virant, F. S., Wald, E. R., Khan, D. A.,
Blessing-Moore, J., Lang, D. M., Nicklas, R. A., Oppenheimer, J. J.,
Portnoy, J. M., Schuller, D. E., Tilles, S. A., Borish, L., Nathan, R.
A., Smart, B. A., \& Vandewalker, M. L. (2005). The diagnosis and
management of sinusitis: {A} practice parameter update. \emph{Journal of
Allergy and Clinical Immunology}, \emph{116}, S13--S47.
\url{https://doi.org/10.1016/j.jaci.2005.09.048}

\bibitem[\citeproctext]{ref-smithDetectionOpium2018}
Smith, R. K., Stacey, R. J., Bergström, E., \& Thomas-Oates, J. (2018).
Detection of opium alkaloids in a {Cypriot} base-ring juglet.
\emph{Analyst}, \emph{143}(21), 5127--5136.
\url{https://doi.org/10.1039/C8AN01040D}

\bibitem[\citeproctext]{ref-sorensenEffectAntioxidants2018}
Sørensen, L. K., \& Hasselstrøm, J. B. (2018). The effect of
antioxidants on the long-term stability of {THC} and related
cannabinoids in sampled whole blood. \emph{Drug Testing and Analysis},
\emph{10}(2), 301--309. \url{https://doi.org/10.1002/dta.2221}

\bibitem[\citeproctext]{ref-sorensenDrugsCalculus2021}
Sørensen, L. K., Hasselstrøm, J. B., Larsen, L. S., \& Bindslev, D. A.
(2021). Entrapment of drugs in dental calculus -- {Detection} validation
based on test results from post-mortem investigations. \emph{Forensic
Science International}, \emph{319}, 110647.
\url{https://doi.org/10.1016/j.forsciint.2020.110647}

\bibitem[\citeproctext]{ref-srdjenovicSimultaneousHPLC2008}
Srdjenovic, B., Djordjevic-Milic, V., Grujic, N., Injac, R., \&
Lepojevic, Z. (2008). Simultaneous {HPLC Determination} of {Caffeine},
{Theobromine}, and {Theophylline} in {Food}, {Drinks}, and {Herbal
Products}. \emph{Journal of Chromatographic Science}, \emph{46}(2),
144--149. \url{https://doi.org/10.1093/chromsci/46.2.144}

\bibitem[\citeproctext]{ref-stavricVariabilityCaffeine1988}
Stavric, B., Klassen, R., Watkinson, B., Karpinski, K., Stapley, R., \&
Fried, P. (1988). Variability in caffeine consumption from coffee and
tea: {Possible} significance for epidemiological studies. \emph{Food and
Chemical Toxicology}, \emph{26}(2), 111--118.
\url{https://doi.org/10.1016/0278-6915(88)90107-X}

\bibitem[\citeproctext]{ref-sunMetabolomicsEvaluation2016}
Sun, J., Jin, J., Beger, R. D., Cerniglia, C. E., Yang, M., \& Chen, H.
(2016). Metabolomics evaluation of the impact of smokeless tobacco
exposure on the oral bacterium {Capnocytophaga} sputigena.
\emph{Toxicology in Vitro}, \emph{36}, 133--141.
\url{https://doi.org/10.1016/j.tiv.2016.07.020}

\bibitem[\citeproctext]{ref-takahashiOralMicrobiome2015}
Takahashi, N. (2015). Oral {Microbiome Metabolism}: {From} {``{Who Are
They}?''} To {``{What Are They Doing}?''} \emph{Journal of Dental
Research}, \emph{94}(12), 1628--1637.
\url{https://doi.org/10.1177/0022034515606045}

\bibitem[\citeproctext]{ref-tushinghamHuntergathererTobacco2013}
Tushingham, S., Ardura, D., Eerkens, J. W., Palazoglu, M., Shahbaz, S.,
\& Fiehn, O. (2013). Hunter-gatherer tobacco smoking: Earliest evidence
from the {Pacific Northwest Coast} of {North America}. \emph{Journal of
Archaeological Science}, \emph{40}(2), 1397--1407.
\url{https://doi.org/10.1016/j.jas.2012.09.019}

\bibitem[\citeproctext]{ref-unoSexAgedependent2017}
Uno, Y., Takata, R., Kito, G., Yamazaki, H., Nakagawa, K., Nakamura, Y.,
Kamataki, T., \& Katagiri, T. (2017). Sex- and age-dependent gene
expression in human liver: {An} implication for drug-metabolizing
enzymes. \emph{Drug Metabolism and Pharmacokinetics}, \emph{32}(1),
100--107. \url{https://doi.org/10.1016/j.dmpk.2016.10.409}

\bibitem[\citeproctext]{ref-valenDetermination212017}
Valen, A., Leere Øiestad, Å. M., Strand, D. H., Skari, R., \& Berg, T.
(2017). Determination of 21 drugs in oral fluid using fully automated
supported liquid extraction and {UHPLC-MS}/{MS}. \emph{Drug Testing and
Analysis}, \emph{9}(5), 808--823. \url{https://doi.org/10.1002/dta.2045}

\bibitem[\citeproctext]{ref-velskoMicrobialDifferences2019}
Velsko, I. M., Fellows Yates, J. A., Aron, F., Hagan, R. W., Frantz, L.
A. F., Loe, L., Martinez, J. B. R., Chaves, E., Gosden, C., Larson, G.,
\& Warinner, C. (2019). Microbial differences between dental plaque and
historic dental calculus are related to oral biofilm maturation stage.
\emph{Microbiome}, \emph{7}(1), 102.
\url{https://doi.org/10.1186/s40168-019-0717-3}

\bibitem[\citeproctext]{ref-velskoDentalCalculus2017}
Velsko, I. M., Overmyer, K. A., Speller, C., Klaus, L., Collins, M. J.,
Loe, L., Frantz, L. A. F., Sankaranarayanan, K., Lewis, C. M., Martinez,
J. B. R., Chaves, E., Coon, J. J., Larson, G., \& Warinner, C. (2017).
The dental calculus metabolome in modern and historic samples.
\emph{Metabolomics}, \emph{13}(11), 134.
\url{https://doi.org/10.1007/s11306-017-1270-3}

\bibitem[\citeproctext]{ref-warinnerEvidenceMilk2014}
Warinner, C., Hendy, J., Speller, C., Cappellini, E., Fischer, R.,
Trachsel, C., Arneborg, J., Lynnerup, N., Craig, O. E., Swallow, D. M.,
Fotakis, A., Christensen, R. J., Olsen, J. V., Liebert, A., Montalva,
N., Fiddyment, S., Charlton, S., Mackie, M., Canci, A., \ldots{}
Collins, M. J. (2014). Direct evidence of milk consumption from ancient
human dental calculus. \emph{Scientific Reports}, \emph{4}, 7104.
\url{https://doi.org/10.1038/srep07104}

\bibitem[\citeproctext]{ref-whiteDentalCalculus1997}
White, D. J. (1997). Dental calculus: Recent insights into occurrence,
formation, prevention, removal and oral health effects of supragingival
and subgingival deposits. \emph{European Journal of Oral Sciences},
\emph{105}(5), 508--522.
\url{https://doi.org/10.1111/j.1600-0722.1997.tb00238.x}

\bibitem[\citeproctext]{ref-ggplot2}
Wickham, H. (2016). \emph{Ggplot2: {Elegant Graphics} for {Data
Analysis}}. {Springer-Verlag}. \url{https://ggplot2.tidyverse.org}

\bibitem[\citeproctext]{ref-tidyverse2019}
Wickham, H., Averick, M., Bryan, J., Chang, W., McGowan, L. D.,
François, R., Grolemund, G., Hayes, A., Henry, L., Hester, J., Kuhn, M.,
Pedersen, T. L., Miller, E., Bache, S. M., Müller, K., Ooms, J.,
Robinson, D., Seidel, D. P., Spinu, V., \ldots{} Yutani, H. (2019).
Welcome to the {tidyverse}. \emph{Journal of Open Source Software},
\emph{4}(43), 1686. \url{https://doi.org/10.21105/joss.01686}

\bibitem[\citeproctext]{ref-willeRelationshipOral2009}
Wille, S. M. R., Raes, E., Lillsunde, P., Gunnar, T., Laloup, M., Samyn,
N., Christophersen, A. S., Moeller, M. R., Hammer, K. P., \& Verstraete,
A. G. (2009). Relationship {Between Oral Fluid} and {Blood
Concentrations} of {Drugs} of {Abuse} in {Drivers Suspected} of {Driving
Under} the {Influence} of {Drugs}. \emph{Therapeutic Drug Monitoring},
\emph{31}(4), 511. \url{https://doi.org/10.1097/FTD.0b013e3181ae46ea}

\bibitem[\citeproctext]{ref-yaussyCalculusSurvivorship2019}
Yaussy, S. L., \& DeWitte, S. N. (2019). Calculus and survivorship in
medieval {London}: {The} association between dental disease and a
demographic measure of general health. \emph{American Journal of
Physical Anthropology}, \emph{168}(3), 552--565.
\url{https://doi.org/10.1002/ajpa.23772}

\bibitem[\citeproctext]{ref-ziesemer16SChallenges2015}
Ziesemer, K. A., Mann, A. E., Sankaranarayanan, K., Schroeder, H., Ozga,
A. T., Brandt, B. W., Zaura, E., Waters-Rist, A., Hoogland, M.,
Salazar-Garcia, D. C., Aldenderfer, M., Speller, C., Hendy, J., Weston,
D. A., MacDonald, S. J., Thomas, G. H., Collins, M. J., Lewis, C. M.,
Hofman, C., \& Warinner, C. (2015). Intrinsic challenges in ancient
microbiome reconstruction using {16S rRNA} gene amplification. \emph{Sci
Rep}, \emph{5}, 16498. \url{https://doi.org/10.1038/srep16498}

\bibitem[\citeproctext]{ref-ziesemerGenomeCalculus2018}
Ziesemer, K. A., Ramos‐Madrigal, J., Mann, A. E., Brandt, B. W.,
Sankaranarayanan, K., Ozga, A. T., Hoogland, M., Hofman, C. A.,
Salazar‐García, D. C., Frohlich, B., Milner, G. R., Stone, A. C.,
Aldenderfer, M., Lewis, C. M., Hofman, C. L., Warinner, C., \&
Schroeder, H. (2018). The efficacy of whole human genome capture on
ancient dental calculus and dentin. \emph{American Journal of Physical
Anthropology}. \url{https://doi.org/10.1002/ajpa.23763}

\bibitem[\citeproctext]{ref-zijngeBiofilmArchitecture2010}
Zijnge, V., van Leeuwen, M. B. M., Degener, J. E., Abbas, F., Thurnheer,
T., Gmür, R., \& M. Harmsen, H. J. (2010). Oral {Biofilm Architecture}
on {Natural Teeth}. \emph{PLoS ONE}, \emph{5}(2), e9321.
\url{https://doi.org/10.1371/journal.pone.0009321}

\end{CSLReferences}



\end{document}
